% Generated by Sphinx.
\def\sphinxdocclass{report}
\documentclass[a4paper,11pt,oneside]{sphinxmanual}
\usepackage[utf8]{inputenc}
\DeclareUnicodeCharacter{00A0}{\nobreakspace}
\usepackage[T1]{fontenc}
\usepackage[english]{babel}
\usepackage{times}
\usepackage[Bjarne]{fncychap}
\usepackage{longtable}
\usepackage{sphinx}
\usepackage{multirow}


\title{pyModis Documentation}
\date{May 06, 2013}
\release{0.6.4}
\author{Luca Delucchi}
\newcommand{\sphinxlogo}{\includegraphics{pyModis.png}\par}
\renewcommand{\releasename}{Release}
\makeindex

\makeatletter
\def\PYG@reset{\let\PYG@it=\relax \let\PYG@bf=\relax%
    \let\PYG@ul=\relax \let\PYG@tc=\relax%
    \let\PYG@bc=\relax \let\PYG@ff=\relax}
\def\PYG@tok#1{\csname PYG@tok@#1\endcsname}
\def\PYG@toks#1+{\ifx\relax#1\empty\else%
    \PYG@tok{#1}\expandafter\PYG@toks\fi}
\def\PYG@do#1{\PYG@bc{\PYG@tc{\PYG@ul{%
    \PYG@it{\PYG@bf{\PYG@ff{#1}}}}}}}
\def\PYG#1#2{\PYG@reset\PYG@toks#1+\relax+\PYG@do{#2}}

\expandafter\def\csname PYG@tok@gd\endcsname{\def\PYG@tc##1{\textcolor[rgb]{0.63,0.00,0.00}{##1}}}
\expandafter\def\csname PYG@tok@gu\endcsname{\let\PYG@bf=\textbf\def\PYG@tc##1{\textcolor[rgb]{0.50,0.00,0.50}{##1}}}
\expandafter\def\csname PYG@tok@gt\endcsname{\def\PYG@tc##1{\textcolor[rgb]{0.00,0.25,0.82}{##1}}}
\expandafter\def\csname PYG@tok@gs\endcsname{\let\PYG@bf=\textbf}
\expandafter\def\csname PYG@tok@gr\endcsname{\def\PYG@tc##1{\textcolor[rgb]{1.00,0.00,0.00}{##1}}}
\expandafter\def\csname PYG@tok@cm\endcsname{\let\PYG@it=\textit\def\PYG@tc##1{\textcolor[rgb]{0.25,0.50,0.56}{##1}}}
\expandafter\def\csname PYG@tok@vg\endcsname{\def\PYG@tc##1{\textcolor[rgb]{0.73,0.38,0.84}{##1}}}
\expandafter\def\csname PYG@tok@m\endcsname{\def\PYG@tc##1{\textcolor[rgb]{0.13,0.50,0.31}{##1}}}
\expandafter\def\csname PYG@tok@mh\endcsname{\def\PYG@tc##1{\textcolor[rgb]{0.13,0.50,0.31}{##1}}}
\expandafter\def\csname PYG@tok@cs\endcsname{\def\PYG@tc##1{\textcolor[rgb]{0.25,0.50,0.56}{##1}}\def\PYG@bc##1{\setlength{\fboxsep}{0pt}\colorbox[rgb]{1.00,0.94,0.94}{\strut ##1}}}
\expandafter\def\csname PYG@tok@ge\endcsname{\let\PYG@it=\textit}
\expandafter\def\csname PYG@tok@vc\endcsname{\def\PYG@tc##1{\textcolor[rgb]{0.73,0.38,0.84}{##1}}}
\expandafter\def\csname PYG@tok@il\endcsname{\def\PYG@tc##1{\textcolor[rgb]{0.13,0.50,0.31}{##1}}}
\expandafter\def\csname PYG@tok@go\endcsname{\def\PYG@tc##1{\textcolor[rgb]{0.19,0.19,0.19}{##1}}}
\expandafter\def\csname PYG@tok@cp\endcsname{\def\PYG@tc##1{\textcolor[rgb]{0.00,0.44,0.13}{##1}}}
\expandafter\def\csname PYG@tok@gi\endcsname{\def\PYG@tc##1{\textcolor[rgb]{0.00,0.63,0.00}{##1}}}
\expandafter\def\csname PYG@tok@gh\endcsname{\let\PYG@bf=\textbf\def\PYG@tc##1{\textcolor[rgb]{0.00,0.00,0.50}{##1}}}
\expandafter\def\csname PYG@tok@ni\endcsname{\let\PYG@bf=\textbf\def\PYG@tc##1{\textcolor[rgb]{0.84,0.33,0.22}{##1}}}
\expandafter\def\csname PYG@tok@nl\endcsname{\let\PYG@bf=\textbf\def\PYG@tc##1{\textcolor[rgb]{0.00,0.13,0.44}{##1}}}
\expandafter\def\csname PYG@tok@nn\endcsname{\let\PYG@bf=\textbf\def\PYG@tc##1{\textcolor[rgb]{0.05,0.52,0.71}{##1}}}
\expandafter\def\csname PYG@tok@no\endcsname{\def\PYG@tc##1{\textcolor[rgb]{0.38,0.68,0.84}{##1}}}
\expandafter\def\csname PYG@tok@na\endcsname{\def\PYG@tc##1{\textcolor[rgb]{0.25,0.44,0.63}{##1}}}
\expandafter\def\csname PYG@tok@nb\endcsname{\def\PYG@tc##1{\textcolor[rgb]{0.00,0.44,0.13}{##1}}}
\expandafter\def\csname PYG@tok@nc\endcsname{\let\PYG@bf=\textbf\def\PYG@tc##1{\textcolor[rgb]{0.05,0.52,0.71}{##1}}}
\expandafter\def\csname PYG@tok@nd\endcsname{\let\PYG@bf=\textbf\def\PYG@tc##1{\textcolor[rgb]{0.33,0.33,0.33}{##1}}}
\expandafter\def\csname PYG@tok@ne\endcsname{\def\PYG@tc##1{\textcolor[rgb]{0.00,0.44,0.13}{##1}}}
\expandafter\def\csname PYG@tok@nf\endcsname{\def\PYG@tc##1{\textcolor[rgb]{0.02,0.16,0.49}{##1}}}
\expandafter\def\csname PYG@tok@si\endcsname{\let\PYG@it=\textit\def\PYG@tc##1{\textcolor[rgb]{0.44,0.63,0.82}{##1}}}
\expandafter\def\csname PYG@tok@s2\endcsname{\def\PYG@tc##1{\textcolor[rgb]{0.25,0.44,0.63}{##1}}}
\expandafter\def\csname PYG@tok@vi\endcsname{\def\PYG@tc##1{\textcolor[rgb]{0.73,0.38,0.84}{##1}}}
\expandafter\def\csname PYG@tok@nt\endcsname{\let\PYG@bf=\textbf\def\PYG@tc##1{\textcolor[rgb]{0.02,0.16,0.45}{##1}}}
\expandafter\def\csname PYG@tok@nv\endcsname{\def\PYG@tc##1{\textcolor[rgb]{0.73,0.38,0.84}{##1}}}
\expandafter\def\csname PYG@tok@s1\endcsname{\def\PYG@tc##1{\textcolor[rgb]{0.25,0.44,0.63}{##1}}}
\expandafter\def\csname PYG@tok@gp\endcsname{\let\PYG@bf=\textbf\def\PYG@tc##1{\textcolor[rgb]{0.78,0.36,0.04}{##1}}}
\expandafter\def\csname PYG@tok@sh\endcsname{\def\PYG@tc##1{\textcolor[rgb]{0.25,0.44,0.63}{##1}}}
\expandafter\def\csname PYG@tok@ow\endcsname{\let\PYG@bf=\textbf\def\PYG@tc##1{\textcolor[rgb]{0.00,0.44,0.13}{##1}}}
\expandafter\def\csname PYG@tok@sx\endcsname{\def\PYG@tc##1{\textcolor[rgb]{0.78,0.36,0.04}{##1}}}
\expandafter\def\csname PYG@tok@bp\endcsname{\def\PYG@tc##1{\textcolor[rgb]{0.00,0.44,0.13}{##1}}}
\expandafter\def\csname PYG@tok@c1\endcsname{\let\PYG@it=\textit\def\PYG@tc##1{\textcolor[rgb]{0.25,0.50,0.56}{##1}}}
\expandafter\def\csname PYG@tok@kc\endcsname{\let\PYG@bf=\textbf\def\PYG@tc##1{\textcolor[rgb]{0.00,0.44,0.13}{##1}}}
\expandafter\def\csname PYG@tok@c\endcsname{\let\PYG@it=\textit\def\PYG@tc##1{\textcolor[rgb]{0.25,0.50,0.56}{##1}}}
\expandafter\def\csname PYG@tok@mf\endcsname{\def\PYG@tc##1{\textcolor[rgb]{0.13,0.50,0.31}{##1}}}
\expandafter\def\csname PYG@tok@err\endcsname{\def\PYG@bc##1{\setlength{\fboxsep}{0pt}\fcolorbox[rgb]{1.00,0.00,0.00}{1,1,1}{\strut ##1}}}
\expandafter\def\csname PYG@tok@kd\endcsname{\let\PYG@bf=\textbf\def\PYG@tc##1{\textcolor[rgb]{0.00,0.44,0.13}{##1}}}
\expandafter\def\csname PYG@tok@ss\endcsname{\def\PYG@tc##1{\textcolor[rgb]{0.32,0.47,0.09}{##1}}}
\expandafter\def\csname PYG@tok@sr\endcsname{\def\PYG@tc##1{\textcolor[rgb]{0.14,0.33,0.53}{##1}}}
\expandafter\def\csname PYG@tok@mo\endcsname{\def\PYG@tc##1{\textcolor[rgb]{0.13,0.50,0.31}{##1}}}
\expandafter\def\csname PYG@tok@mi\endcsname{\def\PYG@tc##1{\textcolor[rgb]{0.13,0.50,0.31}{##1}}}
\expandafter\def\csname PYG@tok@kn\endcsname{\let\PYG@bf=\textbf\def\PYG@tc##1{\textcolor[rgb]{0.00,0.44,0.13}{##1}}}
\expandafter\def\csname PYG@tok@o\endcsname{\def\PYG@tc##1{\textcolor[rgb]{0.40,0.40,0.40}{##1}}}
\expandafter\def\csname PYG@tok@kr\endcsname{\let\PYG@bf=\textbf\def\PYG@tc##1{\textcolor[rgb]{0.00,0.44,0.13}{##1}}}
\expandafter\def\csname PYG@tok@s\endcsname{\def\PYG@tc##1{\textcolor[rgb]{0.25,0.44,0.63}{##1}}}
\expandafter\def\csname PYG@tok@kp\endcsname{\def\PYG@tc##1{\textcolor[rgb]{0.00,0.44,0.13}{##1}}}
\expandafter\def\csname PYG@tok@w\endcsname{\def\PYG@tc##1{\textcolor[rgb]{0.73,0.73,0.73}{##1}}}
\expandafter\def\csname PYG@tok@kt\endcsname{\def\PYG@tc##1{\textcolor[rgb]{0.56,0.13,0.00}{##1}}}
\expandafter\def\csname PYG@tok@sc\endcsname{\def\PYG@tc##1{\textcolor[rgb]{0.25,0.44,0.63}{##1}}}
\expandafter\def\csname PYG@tok@sb\endcsname{\def\PYG@tc##1{\textcolor[rgb]{0.25,0.44,0.63}{##1}}}
\expandafter\def\csname PYG@tok@k\endcsname{\let\PYG@bf=\textbf\def\PYG@tc##1{\textcolor[rgb]{0.00,0.44,0.13}{##1}}}
\expandafter\def\csname PYG@tok@se\endcsname{\let\PYG@bf=\textbf\def\PYG@tc##1{\textcolor[rgb]{0.25,0.44,0.63}{##1}}}
\expandafter\def\csname PYG@tok@sd\endcsname{\let\PYG@it=\textit\def\PYG@tc##1{\textcolor[rgb]{0.25,0.44,0.63}{##1}}}

\def\PYGZbs{\char`\\}
\def\PYGZus{\char`\_}
\def\PYGZob{\char`\{}
\def\PYGZcb{\char`\}}
\def\PYGZca{\char`\^}
\def\PYGZam{\char`\&}
\def\PYGZlt{\char`\<}
\def\PYGZgt{\char`\>}
\def\PYGZsh{\char`\#}
\def\PYGZpc{\char`\%}
\def\PYGZdl{\char`\$}
\def\PYGZti{\char`\~}
% for compatibility with earlier versions
\def\PYGZat{@}
\def\PYGZlb{[}
\def\PYGZrb{]}
\makeatother

\begin{document}

\maketitle
\tableofcontents
\phantomsection\label{index::doc}


\code{pyModis} library was developed to replace old bash scripts developed by Markus
Neteler to download MODIS data from NASA FTP server. It is very useful for
\href{http://gis.cri.fmach.it}{GIS and Remote Sensing Platform} of \href{http://www.fmach.it}{Fondazione Edmund Mach} to update
its large collection of MODIS data.

It has several features:
\begin{itemize}
\item {} 
it is very useful for downloading large numbers of MODIS HDF/XML files
and for using this in a cron job for automated continuous updating

\item {} 
it can parse the XML file to obtain information about the HDF files

\item {} 
it can convert a HDF MODIS file to GEOTIF file using \href{https://lpdaac.usgs.gov/lpdaac/tools/modis\_reprojection\_tool}{MODIS Reprojection Tool}

\item {} 
it can create a mosaic of several tiles using \href{https://lpdaac.usgs.gov/lpdaac/tools/modis\_reprojection\_tool}{MODIS Reprojection Tool} and can
create the xml metadata file with the information of all tiles used in mosaic

\end{itemize}


\chapter{About pyModis}
\label{info:welcome-to-pymodis}\label{info::doc}\label{info:about-pymodis}

\section{Requirements}
\label{info:requirements}
The only required software is \href{https://lpdaac.usgs.gov/tools/modis\_reprojection\_tool}{MODIS Reprojection Tool}
to convert or mosaic MODIS HDF files.

For \emph{download} or \emph{parse} HDF files only standard Python modules are used
by \code{pyModis} library and tools.


\section{How to install pyModis}
\label{info:how-to-install-pymodis}

\subsection{Using pip}
\label{info:using-pip}
From version 0.6.3 it is possible to install \code{pyModis} using
\href{https://pypi.python.org/pypi/pip}{pip}. You have to run the following
command as administrator

\begin{Verbatim}[commandchars=\\\{\}]
pip install pyModis
\end{Verbatim}

If you need to update you \code{pyModis} version you have to run

\begin{Verbatim}[commandchars=\\\{\}]
pip install --upgrade pyModis
\end{Verbatim}

With \code{pip} it is also really simple to remove the library

\begin{Verbatim}[commandchars=\\\{\}]
pip uninstall pyModis
\end{Verbatim}


\subsection{Compile from source}
\label{info:compile-from-source}
Installing \code{pyModis} is very simple. First you need to download \code{pyModis}
source code from \href{https://github.com/lucadelu/pyModis}{github repository}.

You can use \href{http://git-scm.com/}{git} to download the latest code
(with the whole history and so it contain all the different stable versions,
from the last to the first)

\begin{Verbatim}[commandchars=\\\{\}]
git clone https://github.com/lucadelu/pyModis.git
\end{Verbatim}

or \href{https://github.com/lucadelu/pyModis/tags}{download the latest stable version}
from the repository and decompress it.

Now enter the \code{pyModis} folder and launch as administrator of
your computer

\begin{Verbatim}[commandchars=\\\{\}]
python setup.py install
\end{Verbatim}

If the installation doesn't return any errors you should be able to use
\code{pyModis} library from a Python console. Then, launch a your favorite
Python console (I really suggest \code{ipython}) and digit

\begin{Verbatim}[commandchars=\\\{\}]
\PYG{k+kn}{import} \PYG{n+nn}{pymodis}
\end{Verbatim}

If the console doesn't return any error like this

\begin{Verbatim}[commandchars=\\\{\}]
ImportError: No module named pymodis
\end{Verbatim}

the \code{pyModis} library has been installed properly and you can use it
or one of the tools distributed with \code{pyModis}.


\section{How to report a bug}
\label{info:how-to-report-a-bug}
If you find any problems in \code{pyModis} library you can report it using
the \href{https://github.com/lucadelu/pyModis/issues}{issues tracker of github}.


\section{How to compile documentation}
\label{info:how-to-compile-documentation}
This documentation has been made with \href{http://sphinx.pocoo.org}{Sphinx}, so you
need to install it to compile the original files to obtain different
output formats.

Please enter the \code{docs} folder of \code{pyModis} source and run

\begin{Verbatim}[commandchars=\\\{\}]
make \textless{}target\textgreater{}
\end{Verbatim}

with one of the following target to obtain the desired output:
\begin{itemize}
\item {} 
\textbf{html}: to make standalone HTML files

\item {} 
\textbf{dirhtml}: to make HTML files named index.html in directories

\item {} 
\textbf{singlehtml}: to make a single large HTML file

\item {} 
\textbf{pickle}: to make pickle files

\item {} 
\textbf{json}: to make JSON files

\item {} 
\textbf{htmlhelp}: to make HTML files and a HTML help project

\item {} 
\textbf{qthelp}: to make HTML files and a qthelp project

\item {} 
\textbf{devhelp}: to make HTML files and a Devhelp project

\item {} 
\textbf{epub}: to make an epub

\item {} 
\textbf{latex}: to make LaTeX files, you can set PAPER=a4 or PAPER=letter

\item {} 
\textbf{latexpdf}: to make LaTeX files and run them through pdflatex

\item {} 
\textbf{text}: to make text files

\item {} 
\textbf{man}: to make manual pages

\item {} 
\textbf{texinfo}: to make Texinfo files

\item {} 
\textbf{info}: to make Texinfo files and run them through makeinfo

\item {} 
\textbf{gettext}: to make PO message catalogs

\item {} 
\textbf{changes}: to make an overview of all changed/added/deprecated items

\item {} 
\textbf{linkcheck}: to check all external links for integrity

\item {} 
\textbf{doctest}: to run all doctests embedded in the documentation (if enabled)

\end{itemize}


\subsection{PDF link in HTML}
\label{info:pdf-link-in-html}
To insert a link to PDF file of pyModis documentation into HTML documentation
(the link will be added on the sidebar) you have to compile first the PDF and
after the HTML, so you need to launch:

\begin{Verbatim}[commandchars=\\\{\}]
make latexpdf
make html
\end{Verbatim}

If PDF file is missing no link will be added


\section{Ohloh statistics}
\label{info:ohloh-statistics}
For more information about \code{pyModis} please visit the
\href{http://www.ohloh.net/p/pyModis}{pyModis Ohloh page}


\chapter{pyModis Scripts}
\label{scripts/software:pymodis-scripts}\label{scripts/software::doc}
The \code{pyModis} \textbf{scripts} provide you with a complete toolkit to work
with MODIS data, you can download, analyze and convert data. They
are developed to work from command line and inside scripts to
automatically update your MODIS files dataset.

Currently the tools are 5:
\begin{itemize}
\item {} 
{\hyperref[scripts/modis_download::doc]{\emph{modis\_download.py}}}

\item {} 
{\hyperref[scripts/modis_parse::doc]{\emph{modis\_parse.py}}}

\item {} 
{\hyperref[scripts/modis_multiparse::doc]{\emph{modis\_multiparse.py}}}

\item {} 
{\hyperref[scripts/modis_mosaic::doc]{\emph{modis\_mosaic.py}}}

\item {} 
{\hyperref[scripts/modis_convert::doc]{\emph{modis\_convert.py}}}

\end{itemize}
\newpage % hard pagebreak at exactly this position

\section{modis\_download.py}
\label{scripts/modis_download:modis-download-py}\label{scripts/modis_download::doc}
\textbf{modis\_download.py} is a script to download MODIS data from NASA FTP servers. It can download large amounts of data and it can be profitably used with cron jobs to receive data with a fixed delay of time.


\subsection{Usage}
\label{scripts/modis_download:usage}
\begin{Verbatim}[commandchars=\\\{\}]
modis\_download.py [options] destination\_folder
\end{Verbatim}


\subsection{Options}
\label{scripts/modis_download:options}
\begin{Verbatim}[commandchars=\\\{\}]
-h  --help        show the help
-P  --password    password to connect to ftp server,
                  usually your email address  [required]
-U  --username    username to connect to ftp server
                  [default=anonymous]
-u  --url         ftp server url [default=e4ftl01.cr.usgs.gov]
-t  --tiles       string of tiles separated from comma
                  ([default=None] for all tiles)
-s  --source      directory on the ftp
                  ([default=MOLT/MOD11A1.005] for Terra LST data)
-D  --delta       delta of day from the first day [default=10]
-f  --firstday    the day to start download, if you want change
                  data you have to use this format YYYY-MM-DD
                  ([default=None] is for today)
-e  --enddaythe   day to finish download, if you want change
                  data you have to use this format YYYY-MM-DD
                  ([default=None] use delta option)
-x                this is useful for debugging the download
                  [default=False]
-j                download also the jpeg files [default=False]
-O                download only one day, it set delta=1 [default=False]
-A                download all days, it usefull for first download of a
                  product. It overwrite the 'firstday' and 'endday'
                  options [default=False]
-r                remove files with size ugual to zero from
                  'destination\_folder'  [default=False]
\end{Verbatim}


\subsection{Examples}
\label{scripts/modis_download:examples}
Download Terra LST data for a month for Europe

\begin{Verbatim}[commandchars=\\\{\}]
modis\_download.py -P your.mail@prov.org -t TODO -f 2008-01-01 -e 2008-01-31
\end{Verbatim}

Download the last 15 days of Aqua LST data

\begin{Verbatim}[commandchars=\\\{\}]
modis\_download.py -P your.mail@prov.org -s MOLA/MYD11A1.005 -t TODO -D 15
\end{Verbatim}

Download all tiles of NDVI for one day (you have know the rigth day otherwise it download nothing)

\begin{Verbatim}[commandchars=\\\{\}]
modis\_download.py -P your.mail@prov.org -s TODO -f 2010-12-31 -O
\end{Verbatim}
\newpage % hard pagebreak at exactly this position

\section{modis\_parse.py}
\label{scripts/modis_parse:modis-parse-py}\label{scripts/modis_parse::doc}
\textbf{modis\_parse.pys} is a script to parse the XML metadata file for a MODIS
tile and return the requested value. It can also write the metadata information
in a text file.


\subsection{Usage}
\label{scripts/modis_parse:usage}
\begin{Verbatim}[commandchars=\\\{\}]
modis\_parse.py [options] hdf\_file
\end{Verbatim}


\subsection{Options}
\label{scripts/modis_parse:options}
\begin{Verbatim}[commandchars=\\\{\}]
-h  --help     show the help
-a             print all possible values of metadata
-b             print the values related to the spatial max extent
-d             print the values related to the date files
-e             print the values related to the ECSDataGranule
-i             print the input layers
-o             print the other values
-p             print the values related to platform
-q             print the values related to quality
-s             print the values related to psas
-t             print the values related to times
-w  --write    write the chosen information into a file
\end{Verbatim}


\subsection{Examples}
\label{scripts/modis_parse:examples}
Return all values of metadata

\begin{Verbatim}[commandchars=\\\{\}]
modis\_parse.py -a FILE
\end{Verbatim}

Write all values to a file

\begin{Verbatim}[commandchars=\\\{\}]
modis\_parse.py -a -w metadata\_FILE.txt FILE
\end{Verbatim}

Print spatial extent and quality

\begin{Verbatim}[commandchars=\\\{\}]
modis\_parse.py -b -q FILE
\end{Verbatim}
\newpage % hard pagebreak at exactly this position

\section{modis\_multiparse.py}
\label{scripts/modis_multiparse:modis-multiparse-py}\label{scripts/modis_multiparse::doc}
\textbf{modis\_multiparse.py} is a script to parse several XML metadata files
for MODIS tiles. It is very usefull to create XML metadata file for a
mosaic.


\subsection{Usage}
\label{scripts/modis_multiparse:usage}
\begin{Verbatim}[commandchars=\\\{\}]
modis\_multiparse.py [options] hdf\_files\_list
\end{Verbatim}


\subsection{Options}
\label{scripts/modis_multiparse:options}
\begin{Verbatim}[commandchars=\\\{\}]
-h  --help     show the help
-b             print the values related to the spatial max extent
-w  --write    write the MODIS XML metadata file for MODIS mosaic
\end{Verbatim}


\subsection{Examples}
\label{scripts/modis_multiparse:examples}
Print values of spatial bounding box

\begin{Verbatim}[commandchars=\\\{\}]
modis\_multiparse.py -b FILE1 FILE2 ...
\end{Verbatim}

Write xml file to use with hdf file create by {\hyperref[scripts/modis_convert::doc]{\emph{modis\_convert.py}}}

\begin{Verbatim}[commandchars=\\\{\}]
modis\_multiparse.py -w FILE\_mosaic.xml FILE1 FILE2 ...
\end{Verbatim}
\newpage % hard pagebreak at exactly this position

\section{modis\_mosaic.py}
\label{scripts/modis_mosaic:modis-mosaic-py}\label{scripts/modis_mosaic::doc}
\textbf{modis\_mosaic.py} is a script to create a mosaic of several MODIS tiles
in HDF format.


\subsection{Usage}
\label{scripts/modis_mosaic:usage}
\begin{Verbatim}[commandchars=\\\{\}]
modis\_mosaic.py [options] hdflist\_file
\end{Verbatim}


\subsection{Options}
\label{scripts/modis_mosaic:options}
\begin{Verbatim}[commandchars=\\\{\}]
-h  --help      show the help
-m  --mrt       the path to MRT software    [required]
-o  --output    the name of output file    [required]
-s  --subset    a subset of product's layers. The string
                should be similar to: 1 0 [default: all layers]
\end{Verbatim}


\subsection{Examples}
\label{scripts/modis_mosaic:examples}
Convert all the layers of several tiles:

\begin{Verbatim}[commandchars=\\\{\}]
modis\_mosaic.py -m "/usr/local/bin/" -o FILE\_mosaik FILE1 FILE2 ...
\end{Verbatim}

Convert LAYERS of several LST MODIS tiles:

\begin{Verbatim}[commandchars=\\\{\}]
modis\_mosaic.py -s "1 0 1 0" -m "/usr/local/bin/" -o FILE\_mosaik FILE1 FILE2 ...
\end{Verbatim}
\newpage % hard pagebreak at exactly this position

\section{modis\_convert.py}
\label{scripts/modis_convert:modis-convert-py}\label{scripts/modis_convert::doc}
\textbf{modis\_convert.py} is a script to convert MODIS data to TIF formats and
different projection reference system. It is an interface to MRT mrtmosaic
software, the best application for work with HDF MODIS data.


\subsection{Usage}
\label{scripts/modis_convert:usage}
\begin{Verbatim}[commandchars=\\\{\}]
modis\_convert.py [options] hdf\_file
\end{Verbatim}


\subsection{Options}
\label{scripts/modis_convert:options}
\begin{Verbatim}[commandchars=\\\{\}]
-h  --help               show the help
-s  --subset             a subset of product's layers. The string
                         should be similar to: 1 0    [required]
-m  --mrt                the path to MRT software    [required]
-o  --output             the name of output file
-g  --grain              the spatial resolution of output file
-d  --datum              the code of datum
-r  --resampl            the type of resampling
-p  --proj\_parameters    a list of projection parameters
-t  --proj\_type          the output projection system
-u  --utm                the UTM zone if projection system is UTM
\end{Verbatim}

\begin{notice}{warning}{Warning:}
You can find the supported projections in the `Appendix C' of
\href{https://lpdaac.usgs.gov/sites/default/files/public/mrt41\_usermanual\_032811.pdf}{MODIS reprojection tool user's manual} and the datums at section
\code{Datum Conversion} of the same manual
\end{notice}


\subsection{Examples}
\label{scripts/modis_convert:examples}
Convert LAYERS from LST MODIS data with output resolution in 250 meters with
latitude and longitude reference system

\begin{Verbatim}[commandchars=\\\{\}]
modis\_convert.py -s "1 0 1 0" -m "/usr/local/bin/" -g 250 FILE
\end{Verbatim}

Convert LAYERS from NDVI MODIS data with output resolution in 500 meters with
UTM projection in the 32  zone

\begin{Verbatim}[commandchars=\\\{\}]
modis\_convert.py -s "1 0 1 0" -m "/usr/local/bin/" -g 500 -p UTM -u 32 FILE
\end{Verbatim}
\newpage % hard pagebreak at exactly this position

\chapter{Example of a full process}
\label{examples/full_example::doc}\label{examples/full_example:example-of-a-full-process}
In this short example you can understand how to concatenate
the scripts to obtain a GeoTIFF file for each band of the
chosen product.

\begin{notice}{warning}{Warning:}
This example is based on a Linux based system. Please if
you use other OS change the paths where data will be saved
\end{notice}


\section{Downloading data}
\label{examples/full_example:downloading-data}
For first you need to obtain data, so you need to use {\hyperref[scripts/modis_download::doc]{\emph{modis\_download.py}}}

\begin{Verbatim}[commandchars=\\\{\}]
modis\_download.py -f 2012-12-05 -O -t h28v05,h29v05,h28v04
                  -P yourmail@mail.org /tmp/
\end{Verbatim}

\begin{notice}{warning}{Warning:}
In this example we are working on Japan extension, so please
change the name of tiles according with your region.

In this example we download data for only one day (2012-12-05)
using the option ``-O''.

Please change \href{mailto:yourmail@mail.org}{yourmail@mail.org} with your mail.
\end{notice}

Inside \code{/tmp/} directory you will find a file called \emph{listfileMOD11A1.005.txt}
containing the names of files downloaded. The name of file it is related to
the product that you download.

\begin{notice}{warning}{Warning:}
Every time that you download new files of same product it will be overwrite,
so if you need it, you should rename the file
\end{notice}


\section{Mosaic data}
\label{examples/full_example:mosaic-data}
At this point you need to create the mosaic of the tiles downloaded.
{\hyperref[scripts/modis_mosaic::doc]{\emph{modis\_mosaic.py}}} is the script to use.

\begin{Verbatim}[commandchars=\\\{\}]
modis\_mosaic.py -m /path/to/mrt/ -o /tmp/outputfile /tmp/listfileMOD11A1.005.txt
\end{Verbatim}

\begin{notice}{warning}{Warning:}
\code{/path/to/mrt/} is the directory where Modis Reprojection Tools is stored
\end{notice}

The output of this command are \emph{outputfile.hdf} and \emph{outputfile.hdf.xml} inside the
directory \code{/tmp}. It's reading the input files contained in \emph{listfileMOD11A1.005.txt}


\section{Convert data}
\label{examples/full_example:convert-data}
The last part of the procedure is to convert the mosaic, from HDF format and sinusoidal
projection, to GeoTIFF with several projection. You have to use {\hyperref[scripts/modis_convert::doc]{\emph{modis\_convert.py}}}

\begin{Verbatim}[commandchars=\\\{\}]
modis\_convert.py -s '( 1 1 1 1 1 1 1 1 1 1 1 1 )' -m /path/to/mrt/
                 -o /tmp/finalfile.tif -g 250 /tmp/outputfile.hdf
\end{Verbatim}


\chapter{pyModis Library}
\label{pymodis/modules:pymodis-library}\label{pymodis/modules::doc}
\code{pyModis} library it is a Python library to work with MODIS data.

It can easily used in your application to download, analyze and
convert MODIS data, it is already used by GRASS GIS in r.in.modis
addons tools.

It is compose by three module:


\section{pyModis Package}
\label{pymodis/pymodis:pymodis-package}\label{pymodis/pymodis::doc}

\subsection{\texttt{downmodis} module: is very useful to download MODIS data from NASA FTP server}
\label{pymodis/pymodis:downmodis-module-is-very-useful-to-download-modis-data-from-nasa-ftp-server}\label{pymodis/pymodis:module-pymodis.downmodis}\index{pymodis.downmodis (module)}\index{downModis (class in pymodis.downmodis)}

\begin{fulllineitems}
\phantomsection\label{pymodis/pymodis:pymodis.downmodis.downModis}\pysiglinewithargsret{\strong{class }\code{pymodis.downmodis.}\bfcode{downModis}}{\emph{password}, \emph{destinationFolder}, \emph{user='anonymous'}, \emph{url='e4ftl01.cr.usgs.gov'}, \emph{tiles=None}, \emph{path='MOLT/MOD11A1.005'}, \emph{today=None}, \emph{enddate=None}, \emph{delta=10}, \emph{jpg=False}, \emph{debug=False}}{}
A class to download MODIS data from NASA FTP repository
\index{checkDataExist() (pymodis.downmodis.downModis method)}

\begin{fulllineitems}
\phantomsection\label{pymodis/pymodis:pymodis.downmodis.downModis.checkDataExist}\pysiglinewithargsret{\bfcode{checkDataExist}}{\emph{listNewFile}, \emph{move=0}}{}
Check if a file already exists in the directory of download

listNewFile = list of all files, returned by getFilesList function

move = it is useful to know if a function is called from download or move function

\end{fulllineitems}

\index{closeFTP() (pymodis.downmodis.downModis method)}

\begin{fulllineitems}
\phantomsection\label{pymodis/pymodis:pymodis.downmodis.downModis.closeFTP}\pysiglinewithargsret{\bfcode{closeFTP}}{}{}
Close ftp connection

\end{fulllineitems}

\index{connectFTP() (pymodis.downmodis.downModis method)}

\begin{fulllineitems}
\phantomsection\label{pymodis/pymodis:pymodis.downmodis.downModis.connectFTP}\pysiglinewithargsret{\bfcode{connectFTP}}{\emph{ncon=20}}{}
Set connection to ftp server, move to path where data are stored
and create a list of directories for all days

ncon = number maximum of test to connection at the ftp server

\end{fulllineitems}

\index{dayDownload() (pymodis.downmodis.downModis method)}

\begin{fulllineitems}
\phantomsection\label{pymodis/pymodis:pymodis.downmodis.downModis.dayDownload}\pysiglinewithargsret{\bfcode{dayDownload}}{\emph{listFilesDown}}{}
Downloads tiles are in files\_hdf\_consider

listFilesDown = list of the files to download, returned by checkDataExist function

\end{fulllineitems}

\index{debugDays() (pymodis.downmodis.downModis method)}

\begin{fulllineitems}
\phantomsection\label{pymodis/pymodis:pymodis.downmodis.downModis.debugDays}\pysiglinewithargsret{\bfcode{debugDays}}{}{}
This function is useful to debug the number of days

\end{fulllineitems}

\index{debugLog() (pymodis.downmodis.downModis method)}

\begin{fulllineitems}
\phantomsection\label{pymodis/pymodis:pymodis.downmodis.downModis.debugLog}\pysiglinewithargsret{\bfcode{debugLog}}{}{}
Function to create the debug file

\end{fulllineitems}

\index{debugMaps() (pymodis.downmodis.downModis method)}

\begin{fulllineitems}
\phantomsection\label{pymodis/pymodis:pymodis.downmodis.downModis.debugMaps}\pysiglinewithargsret{\bfcode{debugMaps}}{}{}
This function is useful to debug the number of maps to download for
each day

\end{fulllineitems}

\index{downloadsAllDay() (pymodis.downmodis.downModis method)}

\begin{fulllineitems}
\phantomsection\label{pymodis/pymodis:pymodis.downmodis.downModis.downloadsAllDay}\pysiglinewithargsret{\bfcode{downloadsAllDay}}{\emph{clean=False}, \emph{allDays=False}}{}
Downloads all the tiles considered

\end{fulllineitems}

\index{getAllDays() (pymodis.downmodis.downModis method)}

\begin{fulllineitems}
\phantomsection\label{pymodis/pymodis:pymodis.downmodis.downModis.getAllDays}\pysiglinewithargsret{\bfcode{getAllDays}}{}{}
Return a list of all days

\end{fulllineitems}

\index{getFilesList() (pymodis.downmodis.downModis method)}

\begin{fulllineitems}
\phantomsection\label{pymodis/pymodis:pymodis.downmodis.downModis.getFilesList}\pysiglinewithargsret{\bfcode{getFilesList}}{}{}
Create a list of files to download, it is possible choose to download
also the jpeg files or only the hdf files

\end{fulllineitems}

\index{getListDays() (pymodis.downmodis.downModis method)}

\begin{fulllineitems}
\phantomsection\label{pymodis/pymodis:pymodis.downmodis.downModis.getListDays}\pysiglinewithargsret{\bfcode{getListDays}}{}{}
Return a list of all selected days

\end{fulllineitems}

\index{getNewerVersion() (pymodis.downmodis.downModis method)}

\begin{fulllineitems}
\phantomsection\label{pymodis/pymodis:pymodis.downmodis.downModis.getNewerVersion}\pysiglinewithargsret{\bfcode{getNewerVersion}}{\emph{oldFile}, \emph{newFile}}{}
Return newer version of a file

oldFile = one of the two similar file

newFile = one of the two similar file

\end{fulllineitems}

\index{removeEmptyFiles() (pymodis.downmodis.downModis method)}

\begin{fulllineitems}
\phantomsection\label{pymodis/pymodis:pymodis.downmodis.downModis.removeEmptyFiles}\pysiglinewithargsret{\bfcode{removeEmptyFiles}}{}{}
Check if some file has size ugual 0

\end{fulllineitems}

\index{setDirectoryIn() (pymodis.downmodis.downModis method)}

\begin{fulllineitems}
\phantomsection\label{pymodis/pymodis:pymodis.downmodis.downModis.setDirectoryIn}\pysiglinewithargsret{\bfcode{setDirectoryIn}}{\emph{day}}{}
Enter in the directory of the day

\end{fulllineitems}

\index{setDirectoryOver() (pymodis.downmodis.downModis method)}

\begin{fulllineitems}
\phantomsection\label{pymodis/pymodis:pymodis.downmodis.downModis.setDirectoryOver}\pysiglinewithargsret{\bfcode{setDirectoryOver}}{}{}
Come back to old path

\end{fulllineitems}


\end{fulllineitems}



\subsection{\texttt{parsemodis} module: is very simple library to parse MODIS metadata file, it can also write the XML metadata file for a mosaic.}
\label{pymodis/pymodis:parsemodis-module-is-very-simple-library-to-parse-modis-metadata-file-it-can-also-write-the-xml-metadata-file-for-a-mosaic}\label{pymodis/pymodis:module-pymodis.parsemodis}\index{pymodis.parsemodis (module)}\index{parseModis (class in pymodis.parsemodis)}

\begin{fulllineitems}
\phantomsection\label{pymodis/pymodis:pymodis.parsemodis.parseModis}\pysiglinewithargsret{\strong{class }\code{pymodis.parsemodis.}\bfcode{parseModis}}{\emph{filename}}{}
Class to parse MODIS xml files, it also can create the parameter
configuration file for resampling MODIS DATA with the MRT software or
convertmodis Module
\index{confResample() (pymodis.parsemodis.parseModis method)}

\begin{fulllineitems}
\phantomsection\label{pymodis/pymodis:pymodis.parsemodis.parseModis.confResample}\pysiglinewithargsret{\bfcode{confResample}}{\emph{spectral}, \emph{res=None}, \emph{output=None}, \emph{datum='WGS84'}, \emph{resample='NEAREST\_NEIGHBOR'}, \emph{projtype='GEO'}, \emph{utm=None}, \emph{projpar='( 0.0 0.0 0.0 0.0 0.0 0.0 0.0 0.0 0.0 0.0 0.0 0.0 0.0 0.0 0.0 )'}}{}
Create the parameter file to use with resample MRT software to create
tif file

spectral = the spectral subset to be used, look the product table to
understand the layer that you want use. For example:
\begin{itemize}
\item {} 
NDVI ( 1 1 1 0 0 0 0 0 0 0 0 0) copy only layer NDVI, EVI
and QA VI the other layers are not used

\item {} 
LST ( 1 1 0 0 1 1 0 0 0 0 0 0 ) copy only layer daily and
nightly temperature and QA

\end{itemize}

res = the resolution for the output file, it must be set in the map
unit of output projection system. The software will use the
original resolution of input file if res it isn't set

output = the output name, if it doesn't set will use the prefix name
of input hdf file

utm = the UTM zone if projection system is UTM
\begin{description}
\item[{resample = the type of resampling, the valid values are:}] \leavevmode\begin{itemize}
\item {} 
NN (nearest neighbor)

\item {} 
BI (bilinear)

\item {} 
CC (cubic convolution)

\end{itemize}

\item[{projtype = the output projection system, the valid values are:}] \leavevmode\begin{itemize}
\item {} 
AEA (Albers Equal Area)

\item {} 
ER (Equirectangular)

\item {} 
GEO (Geographic Latitude/Longitude)

\item {} 
HAM (Hammer)

\item {} 
ISIN (Integerized Sinusoidal)

\item {} 
IGH (Interrupted Goode Homolosine)

\item {} 
LA (Lambert Azimuthal)

\item {} 
LCC (LambertConformal Conic)

\item {} 
MERCAT (Mercator)

\item {} 
MOL (Mollweide)

\item {} 
PS (Polar Stereographic)

\item {} 
SIN (Sinusoidal)

\item {} 
UTM (Universal TransverseMercator)

\end{itemize}

\item[{datum = the datum to use, the valid values are:}] \leavevmode\begin{itemize}
\item {} 
NAD27

\item {} 
NAD83

\item {} 
WGS66

\item {} 
WGS76

\item {} 
WGS84

\item {} 
NODATUM

\end{itemize}

\end{description}

projpar = a list of projection parameters, for more info check the
Appendix C of MODIS reprojection tool user manual
\href{https://lpdaac.usgs.gov/content/download/4831/22895/file/mrt41\_usermanual\_032811.pdf}{https://lpdaac.usgs.gov/content/download/4831/22895/file/mrt41\_usermanual\_032811.pdf}

\end{fulllineitems}

\index{confResample\_swath() (pymodis.parsemodis.parseModis method)}

\begin{fulllineitems}
\phantomsection\label{pymodis/pymodis:pymodis.parsemodis.parseModis.confResample_swath}\pysiglinewithargsret{\bfcode{confResample\_swath}}{\emph{sds}, \emph{geoloc}, \emph{res}, \emph{output=None}, \emph{sphere=`8'}, \emph{resample='NN'}, \emph{projtype='GEO'}, \emph{utm=None}, \emph{projpar=`0.0 0.0 0.0 0.0 0.0 0.0 0.0 0.0 0.0 0.0 0.0 0.0 0.0 0.0 0.0'}}{}
Create the parameter file to use with resample MRT software to create
tif file

sds = Name of band/s (Science Data Set) to resample

geoloc = Name geolocation file (example MOD3, MYD3)

res = the resolution for the output file, it must be set in the map
unit of output projection system. The software will use the
original resolution of input file if res it isn't set

output = the output name, if it doesn't set will use the prefix name
of input hdf file
\begin{description}
\item[{sphere = Output sphere number. Valid options are:}] \leavevmode\begin{itemize}
\item {} 
0=Clarke 1866

\item {} 
1=Clarke 1880

\item {} 
2=Bessel

\item {} 
3=International 1967

\item {} 
4=International 1909

\item {} 
5=WGS 72

\item {} 
6=Everest

\item {} 
7=WGS 66

\item {} 
8=GRS1980/WGS 84

\item {} 
9=Airy

\item {} 
10=Modified Everest

\item {} 
11=Modified Airy

\item {} 
12=Walbeck

\item {} 
13=Southeast Asia

\item {} 
14=Australian National

\item {} 
15=Krassovsky

\item {} 
16=Hough

\item {} 
17=Mercury1960

\item {} 
18=Modified Mercury1968

\item {} 
19=Sphere 19 (Radius 6370997)

\item {} 
20=MODIS Sphere (Radius 6371007.181)

\end{itemize}

\item[{resample = the type of resampling, the valid values are:}] \leavevmode\begin{itemize}
\item {} 
NN (nearest neighbor)

\item {} 
BI (bilinear)

\item {} 
CC (cubic convolution)

\end{itemize}

\item[{projtype = the output projection system, the valid values are:}] \leavevmode\begin{itemize}
\item {} 
AEA (Albers Equal Area)

\item {} 
ER (Equirectangular)

\item {} 
GEO (Geographic Latitude/Longitude)

\item {} 
HAM (Hammer)

\item {} 
ISIN (Integerized Sinusoidal)

\item {} 
IGH (Interrupted Goode Homolosine)

\item {} 
LA (Lambert Azimuthal)

\item {} 
LCC (LambertConformal Conic)

\item {} 
MERCAT (Mercator)

\item {} 
MOL (Mollweide)

\item {} 
PS (Polar Stereographic),

\item {} 
SIN ()Sinusoidal)

\item {} 
UTM (Universal TransverseMercator)

\end{itemize}

\end{description}

utm = the UTM zone if projection system is UTM

projpar = a list of projection parameters, for more info check
the Appendix C of MODIS reprojection tool user manual
\href{https://lpdaac.usgs.gov/content/download/4831/22895/file/mrt41\_usermanual\_032811.pdf}{https://lpdaac.usgs.gov/content/download/4831/22895/file/mrt41\_usermanual\_032811.pdf}

\end{fulllineitems}

\index{getGranule() (pymodis.parsemodis.parseModis method)}

\begin{fulllineitems}
\phantomsection\label{pymodis/pymodis:pymodis.parsemodis.parseModis.getGranule}\pysiglinewithargsret{\bfcode{getGranule}}{}{}
Set the GranuleURMetaData element

\end{fulllineitems}

\index{getRoot() (pymodis.parsemodis.parseModis method)}

\begin{fulllineitems}
\phantomsection\label{pymodis/pymodis:pymodis.parsemodis.parseModis.getRoot}\pysiglinewithargsret{\bfcode{getRoot}}{}{}
Set the root element

\end{fulllineitems}

\index{retBoundary() (pymodis.parsemodis.parseModis method)}

\begin{fulllineitems}
\phantomsection\label{pymodis/pymodis:pymodis.parsemodis.parseModis.retBoundary}\pysiglinewithargsret{\bfcode{retBoundary}}{}{}
Return the maximum extend (Bounding Box) of the MODIS file as
dictionary

\end{fulllineitems}

\index{retBrowseProduct() (pymodis.parsemodis.parseModis method)}

\begin{fulllineitems}
\phantomsection\label{pymodis/pymodis:pymodis.parsemodis.parseModis.retBrowseProduct}\pysiglinewithargsret{\bfcode{retBrowseProduct}}{}{}
Return the BrowseProduct element

\end{fulllineitems}

\index{retCollectionMetaData() (pymodis.parsemodis.parseModis method)}

\begin{fulllineitems}
\phantomsection\label{pymodis/pymodis:pymodis.parsemodis.parseModis.retCollectionMetaData}\pysiglinewithargsret{\bfcode{retCollectionMetaData}}{}{}
Return the CollectionMetaData element as dictionary

\end{fulllineitems}

\index{retDTD() (pymodis.parsemodis.parseModis method)}

\begin{fulllineitems}
\phantomsection\label{pymodis/pymodis:pymodis.parsemodis.parseModis.retDTD}\pysiglinewithargsret{\bfcode{retDTD}}{}{}
Return the DTDVersion element

\end{fulllineitems}

\index{retDataCenter() (pymodis.parsemodis.parseModis method)}

\begin{fulllineitems}
\phantomsection\label{pymodis/pymodis:pymodis.parsemodis.parseModis.retDataCenter}\pysiglinewithargsret{\bfcode{retDataCenter}}{}{}
Return the DataCenterId element

\end{fulllineitems}

\index{retDataFiles() (pymodis.parsemodis.parseModis method)}

\begin{fulllineitems}
\phantomsection\label{pymodis/pymodis:pymodis.parsemodis.parseModis.retDataFiles}\pysiglinewithargsret{\bfcode{retDataFiles}}{}{}
Return the DataFiles element as dictionary

\end{fulllineitems}

\index{retDataGranule() (pymodis.parsemodis.parseModis method)}

\begin{fulllineitems}
\phantomsection\label{pymodis/pymodis:pymodis.parsemodis.parseModis.retDataGranule}\pysiglinewithargsret{\bfcode{retDataGranule}}{}{}
Return the ECSDataGranule elements as dictionary

\end{fulllineitems}

\index{retDbID() (pymodis.parsemodis.parseModis method)}

\begin{fulllineitems}
\phantomsection\label{pymodis/pymodis:pymodis.parsemodis.parseModis.retDbID}\pysiglinewithargsret{\bfcode{retDbID}}{}{}
Return the DbID element

\end{fulllineitems}

\index{retGranuleUR() (pymodis.parsemodis.parseModis method)}

\begin{fulllineitems}
\phantomsection\label{pymodis/pymodis:pymodis.parsemodis.parseModis.retGranuleUR}\pysiglinewithargsret{\bfcode{retGranuleUR}}{}{}
Return the GranuleUR element

\end{fulllineitems}

\index{retInputGranule() (pymodis.parsemodis.parseModis method)}

\begin{fulllineitems}
\phantomsection\label{pymodis/pymodis:pymodis.parsemodis.parseModis.retInputGranule}\pysiglinewithargsret{\bfcode{retInputGranule}}{}{}
Return the input files (InputGranule) used to process the considered
file

\end{fulllineitems}

\index{retInsertTime() (pymodis.parsemodis.parseModis method)}

\begin{fulllineitems}
\phantomsection\label{pymodis/pymodis:pymodis.parsemodis.parseModis.retInsertTime}\pysiglinewithargsret{\bfcode{retInsertTime}}{}{}
Return the InsertTime element

\end{fulllineitems}

\index{retLastUpdate() (pymodis.parsemodis.parseModis method)}

\begin{fulllineitems}
\phantomsection\label{pymodis/pymodis:pymodis.parsemodis.parseModis.retLastUpdate}\pysiglinewithargsret{\bfcode{retLastUpdate}}{}{}
Return the LastUpdate element

\end{fulllineitems}

\index{retMeasure() (pymodis.parsemodis.parseModis method)}

\begin{fulllineitems}
\phantomsection\label{pymodis/pymodis:pymodis.parsemodis.parseModis.retMeasure}\pysiglinewithargsret{\bfcode{retMeasure}}{}{}
Return statistics of QA as dictionary

\end{fulllineitems}

\index{retPGEVersion() (pymodis.parsemodis.parseModis method)}

\begin{fulllineitems}
\phantomsection\label{pymodis/pymodis:pymodis.parsemodis.parseModis.retPGEVersion}\pysiglinewithargsret{\bfcode{retPGEVersion}}{}{}
Return the PGEVersion element

\end{fulllineitems}

\index{retPSA() (pymodis.parsemodis.parseModis method)}

\begin{fulllineitems}
\phantomsection\label{pymodis/pymodis:pymodis.parsemodis.parseModis.retPSA}\pysiglinewithargsret{\bfcode{retPSA}}{}{}
Return the PSA values as dictionary, the PSAName is the key and
and PSAValue is the value

\end{fulllineitems}

\index{retPlatform() (pymodis.parsemodis.parseModis method)}

\begin{fulllineitems}
\phantomsection\label{pymodis/pymodis:pymodis.parsemodis.parseModis.retPlatform}\pysiglinewithargsret{\bfcode{retPlatform}}{}{}
Return the platform values as dictionary.

\end{fulllineitems}

\index{retRangeTime() (pymodis.parsemodis.parseModis method)}

\begin{fulllineitems}
\phantomsection\label{pymodis/pymodis:pymodis.parsemodis.parseModis.retRangeTime}\pysiglinewithargsret{\bfcode{retRangeTime}}{}{}
Return the RangeDateTime elements as dictionary

\end{fulllineitems}


\end{fulllineitems}

\index{parseModisMulti (class in pymodis.parsemodis)}

\begin{fulllineitems}
\phantomsection\label{pymodis/pymodis:pymodis.parsemodis.parseModisMulti}\pysiglinewithargsret{\strong{class }\code{pymodis.parsemodis.}\bfcode{parseModisMulti}}{\emph{hdflist}}{}
A class to obtain some variables for the xml file of several MODIS tiles.
It can also create the xml file
\index{valBound() (pymodis.parsemodis.parseModisMulti method)}

\begin{fulllineitems}
\phantomsection\label{pymodis/pymodis:pymodis.parsemodis.parseModisMulti.valBound}\pysiglinewithargsret{\bfcode{valBound}}{}{}
Function return the Bounding Box of mosaic

\end{fulllineitems}

\index{valCollectionMetaData() (pymodis.parsemodis.parseModisMulti method)}

\begin{fulllineitems}
\phantomsection\label{pymodis/pymodis:pymodis.parsemodis.parseModisMulti.valCollectionMetaData}\pysiglinewithargsret{\bfcode{valCollectionMetaData}}{\emph{obj}}{}
Function to add CollectionMetaData

obj = element to add CollectionMetaData

\end{fulllineitems}

\index{valDTD() (pymodis.parsemodis.parseModisMulti method)}

\begin{fulllineitems}
\phantomsection\label{pymodis/pymodis:pymodis.parsemodis.parseModisMulti.valDTD}\pysiglinewithargsret{\bfcode{valDTD}}{\emph{obj}}{}
Function to add DTDVersion

obj = element to add DTDVersion

\end{fulllineitems}

\index{valDataCenter() (pymodis.parsemodis.parseModisMulti method)}

\begin{fulllineitems}
\phantomsection\label{pymodis/pymodis:pymodis.parsemodis.parseModisMulti.valDataCenter}\pysiglinewithargsret{\bfcode{valDataCenter}}{\emph{obj}}{}
Function to add DataCenter

obj = element to add DataCenter

\end{fulllineitems}

\index{valDataFiles() (pymodis.parsemodis.parseModisMulti method)}

\begin{fulllineitems}
\phantomsection\label{pymodis/pymodis:pymodis.parsemodis.parseModisMulti.valDataFiles}\pysiglinewithargsret{\bfcode{valDataFiles}}{\emph{obj}}{}
Function to add DataFileContainer

obj = element to add DataFileContainer

\end{fulllineitems}

\index{valDbID() (pymodis.parsemodis.parseModisMulti method)}

\begin{fulllineitems}
\phantomsection\label{pymodis/pymodis:pymodis.parsemodis.parseModisMulti.valDbID}\pysiglinewithargsret{\bfcode{valDbID}}{\emph{obj}}{}
Function to add DbID

obj = element to add DbID

\end{fulllineitems}

\index{valGranuleUR() (pymodis.parsemodis.parseModisMulti method)}

\begin{fulllineitems}
\phantomsection\label{pymodis/pymodis:pymodis.parsemodis.parseModisMulti.valGranuleUR}\pysiglinewithargsret{\bfcode{valGranuleUR}}{\emph{obj}}{}
Function to add GranuleUR

obj = element to add GranuleUR

\end{fulllineitems}

\index{valInputPointer() (pymodis.parsemodis.parseModisMulti method)}

\begin{fulllineitems}
\phantomsection\label{pymodis/pymodis:pymodis.parsemodis.parseModisMulti.valInputPointer}\pysiglinewithargsret{\bfcode{valInputPointer}}{\emph{obj}}{}
Function to add InputPointer

obj = element to add InputPointer

\end{fulllineitems}

\index{valInsTime() (pymodis.parsemodis.parseModisMulti method)}

\begin{fulllineitems}
\phantomsection\label{pymodis/pymodis:pymodis.parsemodis.parseModisMulti.valInsTime}\pysiglinewithargsret{\bfcode{valInsTime}}{\emph{obj}}{}
Function to add the minimum of InsertTime

obj = element to add InsertTime

\end{fulllineitems}

\index{valMeasuredParameter() (pymodis.parsemodis.parseModisMulti method)}

\begin{fulllineitems}
\phantomsection\label{pymodis/pymodis:pymodis.parsemodis.parseModisMulti.valMeasuredParameter}\pysiglinewithargsret{\bfcode{valMeasuredParameter}}{\emph{obj}}{}
Function to add ParameterName

obj = element to add ParameterName

\end{fulllineitems}

\index{valPGEVersion() (pymodis.parsemodis.parseModisMulti method)}

\begin{fulllineitems}
\phantomsection\label{pymodis/pymodis:pymodis.parsemodis.parseModisMulti.valPGEVersion}\pysiglinewithargsret{\bfcode{valPGEVersion}}{\emph{obj}}{}
Function to add PGEVersion

obj = element to add PGEVersion

\end{fulllineitems}

\index{valPlatform() (pymodis.parsemodis.parseModisMulti method)}

\begin{fulllineitems}
\phantomsection\label{pymodis/pymodis:pymodis.parsemodis.parseModisMulti.valPlatform}\pysiglinewithargsret{\bfcode{valPlatform}}{\emph{obj}}{}
Function to add Platform elements

obj = element to add Platform elements

\end{fulllineitems}

\index{valRangeTime() (pymodis.parsemodis.parseModisMulti method)}

\begin{fulllineitems}
\phantomsection\label{pymodis/pymodis:pymodis.parsemodis.parseModisMulti.valRangeTime}\pysiglinewithargsret{\bfcode{valRangeTime}}{\emph{obj}}{}
Function to add RangeDateTime

obj = element to add RangeDateTime

\end{fulllineitems}

\index{writexml() (pymodis.parsemodis.parseModisMulti method)}

\begin{fulllineitems}
\phantomsection\label{pymodis/pymodis:pymodis.parsemodis.parseModisMulti.writexml}\pysiglinewithargsret{\bfcode{writexml}}{\emph{outputname}}{}
Write a xml file for a mosaic

outputname = the name of xml file

\end{fulllineitems}


\end{fulllineitems}



\subsection{\texttt{convertmodis} module: using MRT software, can convert MODIS HDF file to GeoTiff file or create a HDF mosaic file for several tiles.}
\label{pymodis/pymodis:convertmodis-module-using-mrt-software-can-convert-modis-hdf-file-to-geotiff-file-or-create-a-hdf-mosaic-file-for-several-tiles}\label{pymodis/pymodis:module-pymodis.convertmodis}\index{pymodis.convertmodis (module)}\index{convertModis (class in pymodis.convertmodis)}

\begin{fulllineitems}
\phantomsection\label{pymodis/pymodis:pymodis.convertmodis.convertModis}\pysiglinewithargsret{\strong{class }\code{pymodis.convertmodis.}\bfcode{convertModis}}{\emph{hdfname}, \emph{confile}, \emph{mrtpath}}{}
A class to convert modis data from hdf to tif using resample
(from MRT tools)
\index{executable() (pymodis.convertmodis.convertModis method)}

\begin{fulllineitems}
\phantomsection\label{pymodis/pymodis:pymodis.convertmodis.convertModis.executable}\pysiglinewithargsret{\bfcode{executable}}{}{}
Return the executable of resample MRT software

\end{fulllineitems}

\index{run() (pymodis.convertmodis.convertModis method)}

\begin{fulllineitems}
\phantomsection\label{pymodis/pymodis:pymodis.convertmodis.convertModis.run}\pysiglinewithargsret{\bfcode{run}}{}{}
Exec the convertion process

\end{fulllineitems}


\end{fulllineitems}

\index{createMosaic (class in pymodis.convertmodis)}

\begin{fulllineitems}
\phantomsection\label{pymodis/pymodis:pymodis.convertmodis.createMosaic}\pysiglinewithargsret{\strong{class }\code{pymodis.convertmodis.}\bfcode{createMosaic}}{\emph{listfile}, \emph{outprefix}, \emph{mrtpath}, \emph{subset=False}}{}
A class to convert several MODIS tiles into a mosaic
\index{executable() (pymodis.convertmodis.createMosaic method)}

\begin{fulllineitems}
\phantomsection\label{pymodis/pymodis:pymodis.convertmodis.createMosaic.executable}\pysiglinewithargsret{\bfcode{executable}}{}{}
Return the executable of mrtmosaic MRT software

\end{fulllineitems}

\index{run() (pymodis.convertmodis.createMosaic method)}

\begin{fulllineitems}
\phantomsection\label{pymodis/pymodis:pymodis.convertmodis.createMosaic.run}\pysiglinewithargsret{\bfcode{run}}{}{}
Exect the mosaic process

\end{fulllineitems}

\index{write\_mosaic\_xml() (pymodis.convertmodis.createMosaic method)}

\begin{fulllineitems}
\phantomsection\label{pymodis/pymodis:pymodis.convertmodis.createMosaic.write_mosaic_xml}\pysiglinewithargsret{\bfcode{write\_mosaic\_xml}}{}{}
Write the XML metadata file for MODIS mosaic

\end{fulllineitems}


\end{fulllineitems}

\index{processModis (class in pymodis.convertmodis)}

\begin{fulllineitems}
\phantomsection\label{pymodis/pymodis:pymodis.convertmodis.processModis}\pysiglinewithargsret{\strong{class }\code{pymodis.convertmodis.}\bfcode{processModis}}{\emph{hdfname}, \emph{confile}, \emph{mrtpath}, \emph{inputhdf=None}, \emph{outputhdf=None}, \emph{geolocfile=None}}{}
A class to process raw modis data from hdf to tif using swath2grid (from MRT Swath tools)
\index{executable() (pymodis.convertmodis.processModis method)}

\begin{fulllineitems}
\phantomsection\label{pymodis/pymodis:pymodis.convertmodis.processModis.executable}\pysiglinewithargsret{\bfcode{executable}}{}{}
Return the executable of resample MRT software

\end{fulllineitems}

\index{run() (pymodis.convertmodis.processModis method)}

\begin{fulllineitems}
\phantomsection\label{pymodis/pymodis:pymodis.convertmodis.processModis.run}\pysiglinewithargsret{\bfcode{run}}{}{}
Exec the convertion process

\end{fulllineitems}


\end{fulllineitems}


We acknowledge the \href{http://www.fmach.it}{Fondazione Edmund Mach} for promoting the development of
free and open source software.


\renewcommand{\indexname}{Python Module Index}
\begin{theindex}
\def\bigletter#1{{\Large\sffamily#1}\nopagebreak\vspace{1mm}}
\bigletter{p}
\item {\texttt{pymodis.convertmodis}}, \pageref{pymodis/pymodis:module-pymodis.convertmodis}
\item {\texttt{pymodis.downmodis}}, \pageref{pymodis/pymodis:module-pymodis.downmodis}
\item {\texttt{pymodis.parsemodis}}, \pageref{pymodis/pymodis:module-pymodis.parsemodis}
\end{theindex}

\renewcommand{\indexname}{Index}
\printindex
\end{document}
