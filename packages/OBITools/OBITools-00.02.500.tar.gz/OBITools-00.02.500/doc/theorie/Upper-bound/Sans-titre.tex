\documentclass[11pt]{article}
\usepackage{geometry}                % See geometry.pdf to learn the layout options. There are lots.
\geometry{letterpaper}                   % ... or a4paper or a5paper or ... 
%\geometry{landscape}                % Activate for for rotated page geometry
%\usepackage[parfill]{parskip}    % Activate to begin paragraphs with an empty line rather than an indent
\usepackage{graphicx}
\usepackage{amssymb}
\usepackage{amsmath} %maths
\usepackage{epstopdf}
\usepackage[francais]{babel}
\usepackage[latin1]{inputenc}

\DeclareGraphicsRule{.tif}{png}{.png}{`convert #1 `dirname #1`/`basename #1 .tif`.png}

\title{Brief Article}
\author{The Author}
%\date{}                                           % Activate to display a given date or no date

\begin{document}

\section{le probl�me de d�cision}

J'ai deux s�quences et je r�alise un alignement global de type NWS. On calcule le pourcentage d'identit�
comme le nombre de symboles identique entre les deux s�quences rapport� � la longueur de la plus courte s�quence.

avec

\begin{equation}
l_{seq}=\min(length(seq1),length(seq2))
\end{equation}


\begin{equation}
id = \frac{paires identiques}{l_{seq} }
\end{equation}


le nombre de diff�rences $d$ est : 

\begin{equation}
d = \lceil (1-id) \times l_{seq}\rceil
\end{equation}


Si les differences sont �quitablement r�parties on est sur de partager
au moins $d$ mots de taille $l_{max}$

\begin{equation}
l_{max} = \lfloor \frac{l_{seq}}{d+1} - 1\rfloor 
\end{equation}

et il restera un mot de taille $l_{reste}$

\begin{equation}
l_{reste} = l_{seq} - (l_{max}+1) \times d - l_{max}
\end{equation}

soit au minimum en commun $words$ mots de taille $l_{word} \leqslant l_{max}$
\begin{equation}
words = d \times (l_{max} - l_{word} +1) + \max(l_{reste} -l_{word}+1,0)
\end{equation}


M�me si l'on oublie transitoirement le max il nous reste apr�s simplification

\begin{equation}
words = l_{seq} - (d+1) l_{word} - \lfloor \frac{l_{seq}}{d+1}-1 \rfloor +1
\end{equation}

et l� si j'essaye d'isoler d d'un cot�
\begin{equation}
words - l_{seq} + l_{word} - 1 = - d l_{word}  - \lfloor \frac{l_{seq}}{d+1}-1 \rfloor 
\end{equation}

\begin{equation}
l_{seq} -words - l_{word} + 1 =  d l_{word}  + \lfloor \frac{l_{seq}}{d+1}-1 \rfloor 
\end{equation}

je me heurte au floor dans lequel d apparait au d�nominateur et donc je ne sais pas aller plus loin

\begin{equation}
l_{seq} -words - l_{word} + 2 =  d l_{word}  + \lfloor \frac{l_{seq}}{d+1} \rfloor 
\end{equation}

\end{document}  